\documentclass[a4paper,11pt]{article}
\usepackage[square,numbers]{natbib}
\bibliographystyle{unsrtnat}

%%%%% Idioma y codificación
\usepackage[spanish]{babel}
\addto\captionsspanish{%
    \def\tablename{Tabla}%
}
%\usepackage{ucs}
\usepackage[utf8]{inputenc}

\usepackage[T1]{fontenc}
\usepackage{lmodern}
\usepackage{enumitem}
\usepackage{tikz}

%%%%% Margins
\topmargin=-0.45in
\evensidemargin=0in
\oddsidemargin=0in
\headsep=0.25in
\textwidth=450pt
\textheight=660pt
\linespread{1.0} % Line spacing
\setlength{\parskip}{1em}

% Paquete comentarios
\usepackage{comment}

\usepackage{soul} % para tachar palabras
\usepackage[normalem]{ulem}

%%%%% URLs
\usepackage{url} % para insertar urls
\usepackage{multirow} % para las tablas
\usepackage{hyperref}
\usepackage{graphicx} % Required to insert images
\usepackage{subcaption}
\usepackage{float} % Imagenes
\usepackage{placeins}
\usepackage{array}
%%
\makeatletter
\g@addto@macro{\UrlBreaks}{\UrlOrds}
\makeatother
%\usepackage{breakurl}
%\usepackage[breaklinks]{hyperref}
%\def\UrlBreaks{\do\/\do-}

\newfloat{lstfloat}{htbp}{lop}
\floatname{lstfloat}{Código}
\def\lstfloatautorefname{Código}

%%%%% Figuras y punto en vez de dos puntos
\usepackage{wrapfig}
\usepackage{caption}
\DeclareCaptionLabelSeparator{punto}{. }
\captionsetup{labelsep=punto}

%\usepackage{subcaption}
\usepackage{lscape}
\usepackage{courier} % Required for the courier font
\usepackage{lipsum} % Used for inserting dummy 'Lorem ipsum' text into the template%
%\usepackage[usenames,dvipsnames]{color} % Required for custom colors
\usepackage{xcolor}
\usepackage{hyperref}
\hypersetup{
    unicode=false,          % non-Latin characters IN Acrobat’s bookmarks
    pdftoolbar=true,        % show Acrobat’s toolbar?
    pdfmenubar=true,        % show Acrobat’s menu?
    pdffitwindow=false,     % window fit to page when opened
    pdfstartview={FitH},    % fits the width of the page to the window
    pdftitle={My title},    % title
    pdfauthor={Author},     % author
    pdfsubject={Subject},   % subject of the document
    pdfcreator={Creator},   % creator of the document
    pdfproducer={Producer}, % producer of the document
    pdfkeywords={keyword1} {key2} {key3}, % list of keywords
    pdfnewwindow=true,      % links IN new PDF window
    colorlinks=true,        % false: boxed links; true: colored links
    linkcolor=black,          % color of internal links (change box color with linkbordercolor)
    citecolor=blue,        % color of links to bibliography
    filecolor=magenta,      % color of file links
    urlcolor=blue
}

\usepackage{color}
\usepackage{cancel}
\usepackage{mathtools}

%%Codigo de consola
\usepackage{verbatim}
\usepackage{mwe}
\usepackage{listings}
\usepackage{etoolbox}
\usepackage{amsmath}

\definecolor{mygreen}{rgb}{0,0.6,0}
\definecolor{mygray}{rgb}{0.47,0.47,0.33}
\definecolor{myorange}{rgb}{0.8,0.4,0}
\definecolor{mywhite}{rgb}{0.98,0.98,0.98}
\definecolor{myblue}{rgb}{0.01,0.61,0.98}

\newcommand*{\FormatDigit}[1]{\ttfamily\textcolor{mygreen}{#1}}
%% http://tex.stackexchange.com/questions/32174/listings-package-how-can-i-format-all-numbers
\lstdefinestyle{FormattedNumber}{%
    literate=*{0}{{\FormatDigit{0}}}{1}%
        {1}{{\FormatDigit{1}}}{1}%
        {2}{{\FormatDigit{2}}}{1}%
        {3}{{\FormatDigit{3}}}{1}%
        {4}{{\FormatDigit{4}}}{1}%
        {5}{{\FormatDigit{5}}}{1}%
        {6}{{\FormatDigit{6}}}{1}%
        {7}{{\FormatDigit{7}}}{1}%
        {8}{{\FormatDigit{8}}}{1}%
        {9}{{\FormatDigit{9}}}{1}%
        {.0}{{\FormatDigit{.0}}}{2}% Following is to ensure that only periods
        {.1}{{\FormatDigit{.1}}}{2}% followed by a digit are changed.
        {.2}{{\FormatDigit{.2}}}{2}%
        {.3}{{\FormatDigit{.3}}}{2}%
        {.4}{{\FormatDigit{.4}}}{2}%
        {.5}{{\FormatDigit{.5}}}{2}%
        {.6}{{\FormatDigit{.6}}}{2}%
        {.7}{{\FormatDigit{.7}}}{2}%
        {.8}{{\FormatDigit{.8}}}{2}%
        {.9}{{\FormatDigit{.9}}}{2}%
        %{,}{{\FormatDigit{,}}{1}% depends if you want the "," in color
        {\ }{{ }}{1}% handle the space
    ,%
}

\lstset{
    frame = single,
    breaklines=true,
    postbreak=\mbox{\textcolor{red}{$\hookrightarrow$}\space},
}

%%%% Encabezado y pie de pagina
\usepackage{fancyhdr}
\pagestyle{fancy}
\lhead{}
\rhead{\nouppercase{\rightmark}}
\cfoot{\thepage}

\renewcommand{\headrulewidth}{0.4pt} % grosor de la línea de la cabecera
\renewcommand{\footrulewidth}{0.4pt} % grosor de la línea del pie

\setlength{\headheight}{15pt}

\usepackage{etoolbox}
\let\bbordermatrix\bordermatrix
\patchcmd{\bbordermatrix}{8.75}{4.75}{}{}
\patchcmd{\bbordermatrix}{\left(}{\left[}{}{}
    \patchcmd{\bbordermatrix}{\right)}{\right]}{}{}

%\setcounter{secnumdepth}{3}

