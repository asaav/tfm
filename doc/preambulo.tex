\section*{Agradecimientos}

El desarrollo de este trabajo de fin de máster ha supuesto un gran reto académico y profesional, ampliando la ambición que tenía en su día mi trabajo de fin de grado y empleando técnicas que nunca había puesto en práctica en entornos reales. He trabajado durante 8 meses para llevarlo hasta su finalización, debido a la carga de trabajo del máster y la dificultad añadida que ha supuesto de cara a la coordinación con los tutores la situación derivada de la pandemia en la que vivimos.

Quiero agradecer en primer lugar la siempre imprescindible ayuda de ambos tutores, Pedro Enrique López de Teruel Alcolea y Lorenzo Fernández Maimó. Sin su saber, paciencia y consejos, este trabajo jamás podría haber llegado a este estado.

Aunque han colaborado indirectamente, me gustaría agradecer al Ayuntamiento de Molina de Segura que permitiera en su día instalar la cámara de la cual se han extraído las imágenes que se han usado en el desarrollo de este trabajo, al Club Molina Volley por prestarse y a los grupos de investigación de Computación Móvil y Visión Artificial (al que pertenecen los directores del proyecto), MoVi, así como al grupo de investigación Human Movement and Sport Science, HUMSE, por llevar la parte logística. Sin estas imágenes, la obtención del dataset con el que se ha entrenado el modelo final no habría sido posible.

Por último y no menos importante, me gustaría reconocer a mis familiares, amigos y pareja su apoyo incondicional en los buenos momentos y, sobre todo, en aquellos en que más parecía que se estancaba el trabajo. Gracias a su apoyo seguí adelante dando lo mejor de mí mismo.

A todas estas personas, que han hecho posible que este trabajo haya llegado a ser lo que es, les doy mi mas sincero agradecimiento.
\newpage
\section*{Declaración de autenticidad}

D. Antonio Saavedra Sánchez, con DNI ----, estudiante de la titulación de Máster Universitario en Tecnologías de Análisis de Datos Masivos: Big Data y autor del TFM titulado ``Técnicas de detección y seguimiento visual de objetos mediante deep learning''. De acuerdo con el Reglamento por el que se regulan los Trabajos Fin de Grado y de Fin de Máster en la Universidad de Murcia, así como la normativa interna para la oferta, asignación, elaboración y defensa de los Trabajos Fin de Grado y Fin de Máster de las titulaciones impartidas en la Facultad de Informática de la Universidad de Murcia

DECLARO:

Que el Trabajo Fin de Máster presentado para su evaluación es original y de elaboración personal. Todas las fuentes utilizadas han sido debidamente citadas. Así mismo, declaro que no incumple ningún contrato de confidencialidad, ni viola ningún derecho de propiedad intelectual e industrial.

\begin{center}
    Murcia, a 8 de septiembre de 2020

    \includegraphics[width=0.3\textwidth]{./images/firma}

    Fdo.: Antonio Saavedra Sánchez \\
    Autor del TFM

\end{center}





\newpage